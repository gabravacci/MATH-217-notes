\documentclass[twocolumn,10pt]{article}

\usepackage[utf8]{inputenc}
\usepackage[T1]{fontenc}
\usepackage{textcomp}
\usepackage{enumitem}
\usepackage{subcaption}

\usepackage{amsmath, amsfonts, mathtools, amsthm, amssymb}
\usepackage{bm}

\usepackage[margin=0.7in]{geometry}

\usepackage{newpxtext, newpxmath}
%\usepackage{newtxtext,newtxmath}
%\usepackage{iwona}
%\usepackage{cmbright}

\begin{document}
\thispagestyle{empty}
\section{Differential Forms}	
In $\mathbb{R}^{3}$ differential forms can be defined as:	
\begin{itemize}
	\item A 0-form is a function $f(x,y,z)$.
	\item A 1-form is of the form
		\[
			F_1\mathrm{d}x + F_2\mathrm{d}y + F_3\mathrm{d}z
		\] 
	\item A 2-form is of the form
		\[
			F_1\mathrm{d}y\wedge \mathrm{d}z + F_2\mathrm{d}z \wedge \mathrm{d}x + F_3\mathrm{d}x\wedge \mathrm{d}y
		\] 
	\item A 3-form is of the form
		\[
			f(x,y,z)\mathrm{d}x\wedge \mathrm{d}y\wedge \mathrm{d}z
		\] 
\end{itemize}
Differential forms behave somewhat as expected with addition.
\subsection{Multiplication of Differential Forms}
Multiplication of differential forms is \textbf{noncommutative} (but it is associative). Multiplication by 0-forms (i.e. functions) behaves as you would expect and is usually written without the $\wedge$.\\
Let $\omega$ be a $k$-form and $\omega'$ be a $k'$-form. Their product is a $(k+k')$-form and is denoted by $\omega \wedge \omega'$. It behaves as follows:
\begin{enumerate}[label=\alph*.]
	\item $\omega \wedge \omega'$ is linear. That is, $\omega = f_1\omega_1 + f_2\omega_2$ then
		\[
			(f_1\omega_1 + f_2\omega_2)\wedge \omega' = f_1(\omega_1\wedge \omega') + f_2(\omega_2\wedge \omega')
		\] 
	\item $\omega \wedge \omega' = (-1)^{kk'}\omega' \wedge \omega$ 
	\item $\omega \wedge \omega = 0$
\end{enumerate}
\subsubsection{Addendum: The Hodge Star operator}
We define the Hodge Star ($*$) operation on a form $\omega$ of $k\geq 2$in $\mathbb{R}^{n}$ as follows
\[
	\omega \wedge *\omega = \Omega_n 
\] 
where $\Omega_n$ is the complete $n=k$-form. An example makes this clearer. In $\mathbb{R}^{3}$ :
\begin{align*}
	*\mathrm{d} x &= \mathrm{d} y \wedge \mathrm{d} z	
\end{align*}
\subsection{Differentiation of Differential Forms}
One of the main reasons differential forms are so useful is that they very nicely with differentiation. Let $\omega$ be a $k$-form, then 
\[
	\mathrm{d}:= \omega_k \to \omega_{k+1}
\] 
Still working within the scope of $\mathbb{R}^{3}$, consider a $0$-form $f $. Then
\[
	\mathrm{d}f = \frac{\partial f}{\partial x} \mathrm{d}x + \frac{\partial f}{\partial y} \mathrm{d}y + \frac{\partial f}{\partial z} \mathrm{d}z
\] 
Is the equivalent $(k+1)$-form after differentation. Note that this is an edge case, for all other $k$-forms, the differential operator behaves as follows:
\begin{enumerate}
	\item Apply $\mathrm{d}$ to coefficients;
	\item Expand with the wedge product.
\end{enumerate}
\subsubsection{Example:}
Let $\omega$ be a $1$-form. Then
\begin{align*}
	\omega &= P\mathrm{d}x + Q\mathrm{d}y + R \mathrm{d}z\\
	\mathrm{d} \omega &= \mathrm{d} P\wedge \mathrm{d} x + \mathrm{d} Q \wedge \mathrm{d} y + \mathrm{d} R\wedge \mathrm{d} z\\
					  &= (P_x \mathrm{d} x + P_y \mathrm{d} y + P_z \mathrm{d} z)\wedge \mathrm{d} x +\\
					  &+ (Q_x \mathrm{d} x + Q_y \mathrm{d} y + Q_z \mathrm{d} z)\wedge \mathrm{y}  + \\
					  &+ (R_x \mathrm{d} x + R_y \mathrm{d} y + R_z \mathrm{d} z)\wedge \mathrm{d} z =\\
					  &= P_y \mathrm{d} y \wedge \mathrm{d} x + P_z \mathrm{d} z \wedge \mathrm{d} x + \\
					  &+ Q_x \mathrm{d} x \wedge \mathrm{d} y + Q_z \mathrm{d} z \wedge \mathrm{d} y +\\
					  &+ R_x \mathrm{d} x \wedge \mathrm{d} z + R_y \mathrm{d} y \wedge \mathrm{d} z =\\
					  &= (R_y - Q_z)\mathrm{d} y \wedge \mathrm{d} z + (P_z - R_x) \mathrm{d} z \wedge \mathrm{d} x + (Q_x - P_y)\mathrm{d} x \wedge \mathrm{d}y
\end{align*}
Which corresponds to the \textbf{curl} operation.
\subsection{Integration of Differential Forms}
A $k$-form may be integrated over a $k$-dimensional manifold. Or, in simpler terms: We may integrate $k$ times a $k$-form.
For example, we integrate a $1$-form over a $1$-manifold $\mathcal{C}$ (curve):
\[
	\int_{\mathcal{C}} P \mathrm{d} x + Q \mathrm{d} y + R \mathrm{d} z
\] 
Similarly, the integral of a $3$-form over a $3$-manifold (solid) is just:
\[
	\iiint_\mathcal{V} f \mathrm{d} x \wedge \mathrm{d}y \wedge \mathrm{d} z  
\] 
Where the $\wedge$ 's are usually omitted.
\subsubsection{Parametrization}
Consider a parametrization of a surface $\mathcal{S}$ :
\[
\mathbf{r} (u,v) = \left\langle x(u,v),y(u,v),z(u,v) \right\rangle 
\] 
Then we can change the basis of our differential forms using the definition of the differential operator:
\[
\mathrm{d} x = \frac{\partial x}{\partial u} \mathrm{d} u + \frac{\partial x}{\partial \mathbf{v} } \mathrm{d} v = x_u \mathrm{d} u + x_v \mathrm{d} v
\] 
and similarly for $y,z$. Then for $2$-form $\omega$ defined as 
\[
\omega = P \mathrm{d} y \wedge \mathrm{d} z + Q \mathrm{d} z \wedge \mathrm{d} x + R \mathrm{d} x \wedge \mathrm{d} y
\] 
notice that 
\begin{align*}
	\mathrm{d} x \wedge \mathrm{d} y &= (x_u \mathrm{d} u + x_v \mathrm{d} v) \wedge (y_u \mathrm{d} u + y_v \mathrm{d} v)\\
									 &= (x_u y_v - x_v y_u)\mathrm{d} u \wedge \mathrm{d} v
\end{align*}
\subsection{Generalized Stoke's Theorem}
Let $\omega$ be a $k$-form and $\Omega$ be a $(k+1)$ manifold. Then
\[
	\int_\Omega d\omega = \int_{\partial \Omega} \omega
\] 
\end{document}

