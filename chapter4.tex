\chapter{Integral Theorems}
\section{Divergence Theorem}
First, some definitions:
\begin{definition}
	\begin{enumerate}[label=\alph*.]
		\item A surface is \textbf{smooth} if it has a parametrization $\mathbf{r} (u,v)$ with continuous partial derivative $\frac{\partial \mathbf{r} }{\partial u}$ and $\frac{\partial \mathbf{r} }{\partial \mathbf{v} } $ and their cross product is nonzeron.
		\item A surface is \textbf{piecewise smooth} if it is composed of multiple smooth surfaces.
		
	\end{enumerate}
\end{definition}
which leads us into
\begin{theorem}[Divergence Theorem]
	Let $V$ be a bounded solid with a piecewise smooth surface $\partial V$ and let $\mathbf{F} $ be a vector field that has continous first partial derivatives in $V$. Then
	\[
		\iint_{\partial V} \mathbf{F} \cdot \hat{\mathbf{n}}\mathrm{d}S
		=
		\iiint_V \bm\nabla \cdot \mathbf{F} \mathrm{d}V
	\] 
\end{theorem}


\section{Green's Theorem}
First we need some definitions:
\begin{definition}
	A curve $C$ with parametrization $\mathbf{r}(t), a \leq t \leq b$, is \textbf{closed} if $\mathbf{r}(a)= \mathbf{r}(b)$.
\end{definition}
\begin{definition}
	A curve $C$ is \textbf{simple} if it does not cross itself.
\end{definition}
\begin{definition}
	A curve $C$ is \textbf{piecewise smooth} if it has a parametrization $\mathbf{r}(t)$ which is continuous, differentiable and the derivative is also continous and nonzero.
\end{definition}
\begin{theorem}[Green's Theorem]
	Let $R$ be a finite region in the $xy$-plane and let $C$ bound $R$ and consist of finite number of simple, closed and piecewise smooth curves that are oriented consitently with $R$. Then let $F_1$ and $F_2$ have continuous first partial derivative in $R$. Then
	\[
		\oint_C \left[ F_1(x,y)\mathrm{d}x + F_2(x,y)\mathrm{d}y \right]  = \iint_R \left( \frac{\partial F_2}{\partial x} - \frac{\partial F_1}{\partial y}  \right) \mathrm{d}x\mathrm{d}y
	\]
	it is often more useful to define it as
	\[
		\oint_C \left\langle P, Q \right\rangle \cdot \mathrm{d}\mathbf{r} = \iint_R \left(  Q_x - P_y \right)\mathrm{d}x\mathrm{d}y
	\] 
\end{theorem}
\begin{corollary}
	Consider a region $R$ and a curve $C$ such that Green's Theorem applies, then
	 \[
		 \operatorname{Area}(R) = \frac{1}{2}\oint_C \left[ x\mathrm{d}y - y\mathrm{d}x\right] 
	 \] 
\end{corollary}
\begin{eg}
	Let us compute the area of the circle $x^{2}+y^2 \leq a^{2}$ using Green's Theorem. We parametrize it as
	\[
		\mathbf{r}(t) = a\cos t \ihat + a\sin t \jhat
	\] 
	with $0\leq t \leq 2\pi$. Then
	\begin{align*}
		A &= \iint_R \mathrm{d}x\mathrm{d}y\\&=  \frac{1}{2}\oint_C \left[ x\mathrm{d}y - y\mathrm{d}x \right]\\
		  &= \frac{1}{2} \int_{0}^{2\pi} (a\cos t)(a \cos t) - (a\sin t)(-a\sin t) \mathrm{d}t\\
		  &=\frac{1}{2} \int_{0}^{2\pi} a^{2} \cos^{2} t + a^{2}\sin^{2}t \mathrm{d}t\\
		  &=\frac{1}{2} \int_{0}^{2\pi} a^{2} \mathrm{d}t\\
		  &= \frac{1}{2} a^{2} \int_{0}^{2\pi}  \mathrm{d}t\\
		  &= \frac{1}{2}a^{2}(2\pi)\\
		  &= \pi a^{2}
	\end{align*}
\end{eg}
\section{Stoke's Theorem}
\begin{definition}[Stoke's Theorem]		
Let $S$ be a piecewise smooth oriented surface whose boundary ($\partial S$) also consits of a finite number of piecewise, smooth, simple curves oriented consitently with $\hat{\mathbf{n} }$.	Such that if you walk along $\partial S$, $\hat{\mathbf{n}}$ points upwards and $S$ is on your left. And let $\mathbf{F} $ be a vector field that has continuous first partial derivatives in $S$. Then
\[
	\oint_{\partial S} \mathbf{F} \cdot \mathrm{d}\mathbf{r} = \iint_S \bm{\nabla} \times \mathbf{F} \cdot \hat{\mathbf{n} }\mathrm{d}S
\] 


\end{definition}

